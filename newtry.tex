A-life and AI have been two of the great unfulfilled promises of the computing age. A key component in any potential unified theory of complex adaptive systems, the twin disciplines have experienced a long winter of disillusionment and lack of visibility since the glory days of the 1960s. However, the field has not been completely dormant, and incremental progress has been made on numerous fronts, including synthetic evolution, cognitive architecture, and other areas.

Artificial life is probably one of the most-sought after yet unfulfilled
promises of the computing era. What exactly makes the search for A-life and
open-ended evolution an important part of the search for a general theory
of cas? There have been a number of novel attempts to create  workable
a-life, or simulate open-ended biological evolution, included Tierra,
Darwin @ Home, Spore, and many others. Somehow, all of these have failed.
Why have they failed, and what will be necessary for a true solution to the
problem to finally succeed?



As the computer sciences have advanced, we have seen an increase in predictive, adaptive, and learning technologies. However, most of these changes have been hidden from the average user and have become fairly ubiquitos.


Tierra was an open-ended experiment by ecologist Thomas Ray. During the 1980s, a programmer's game called Core Wars enjoyed a small degree of popularity. Two programmers write competing programs that run in a 2D-dimensional grid, also known as a virtual arena or a "core". A programmer's instructions A programmer's goal is to have their own program be the last one standing.

There are ten instructions, and every instruction is a fixed-size three-tuple of a source address, destination address, and the opcode itself. Aside from basic arithmetic, most redcode instructions allow self-modification and self-copying. In Core Wars, polymorphism is the rule rather than the exception. Instructions are indivisible, and there is no such thing as an invalid instruction. A very small number of instructions here can lead to a surprising degree of complexity, and there are instructions for more esoteric purposes, such as simulated parallel programming.

Thomas Ray took these basic concepts and altered them, making Tierra:
	1) Cloning operations would randomly mutate
	2) We had multiple distinct programs that would "compete" for survival, instead of a single program
	3) A background "reaver" killed off programs that were deemed too old.
	
The initial grid was populated with a very simple 80-byte large program that just duplicated itself. After a few thousand clones, the mutation rate caused small variants to appear that were largely identical. However, after some generations, vastly different programs were "evolved", including parasitic programs that were only 45 bytes long, even shorter programs that were parasites of the parasites, and so forth.	

Will having three dimensions in a simulation be necessary? For a structural simulation of life, obviously. But for a behaviorial simulation, perhaps not - we can find abundant diversity in the numbers of bacteria, protozoa, and other microorganisms, and we can say that these coexist on an approximately two-dimensional plane - right? Well, we don't know. But it's a point worth considering.

One thing to note in many of these simulations is that there is an *explicit* way to reproduce or breed. Would it be fruitful to devise a system in which 


[1] Beginners Guide to Redcode http://vyznev.net/corewar/guide.html
http://devolab.msu.edu/
http://www.infidels.org/library/modern/meta/getalife/coretierra.html
http://www.biota.org/book/chbi/chbi4.htm